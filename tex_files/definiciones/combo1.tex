\documentclass{article}
\usepackage[utf8]{inputenc}
\usepackage{amsfonts}
\usepackage{amsmath}
\title{Lenguajes Formales y Computabilidad \\
        \large Definiciones y Convenciones: Combo 1 }

\author{Nicolás Cagliero}


\begin{document}
\maketitle

\begin{enumerate}
    \item Defina cuando un conjunto $S \subseteq \omega^n \times \Sigma^{*m}$ 
    es llamado $\Sigma$-recursivo (no hace falta que defina "función $\Sigma$-recursiva")

    \item Defina $\langle s_1, s_2, \dots \rangle$
    
    \item Defina "$f$ es una función $\Sigma$-mixta"
    
    \item Defina "familia $\Sigma$-indexada de funciones"
    \item Defina $R(f, \mathcal{G})$ (haga el caso de valores numéricos)
\end{enumerate}
\(\)
\begin{center}
    Respuestas: 
    \(\)
\end{center}

\begin{enumerate}
    \item Un conjunto $S \subseteq \omega^n \times \Sigma^{*m}$ 
    es llamado $\Sigma$-recursivo cuando la función $\chi_{S}^{\omega^n \times \Sigma^{*m}}$
    sea $\Sigma$-recursiva
    \item $\langle s_1, s_2, \dots \rangle$ es lo que usamos para denotar al número
    $\displaystyle \prod_{i=1}^{\infty} pr(i)^{s_i}$ dada una infinitupla 
    $(s_1, s_2, \dots) \in \omega^{[N]}$
    \item Sea $\Sigma$ un alfabeto finito. Dada una función $f$, diremos que $f$ es 
    $\Sigma-mixta$ si cumple las siguientes propiedades:
    
        (M1) Existen $n, m \geq 0$, tales que $D_f \subseteq \omega^n \times \Sigma^{*m}$\\
        (M2) Ya sea $I_f \subseteq \omega$ o $I_f \subseteq \Sigma^{m}$
    
    \item Dado un alfabeto $\Sigma$, una familia $\Sigma$-indexada de funciones es una función $\mathcal{G}$ tal que $D_{\mathcal{G}} = \Sigma$ y para cada
    $a \in D_{\mathcal{G}}$ se tiene que $D_{\mathcal{G}}(a)$ es una función.
    \item Sea
    \begin{center}
        $f : S_1 \times \dots \times S_n \times L_1 \times \dots \times L_m \rightarrow \omega$ 
    \end{center}
    con $S_1 \times \dots \times S_n \subseteq \omega$ y $L_1 \times \dots \times L_m \subseteq \Sigma^{*}$
    conjuntos no vacíos y sea $\mathcal{G}$ una familia $\Sigma$-indexada de funciones
    tal que 
    \begin{center}
        $\mathcal{G}_a : \omega \times S_1 \times \dots \times S_n \times L_1 \times \dots \times L_m \rightarrow \omega$
    \end{center}
    para cada $a \in \Sigma$, definimos
    \begin{center}
        $R(f, \mathcal{G}) : S_1 \times \dots \times S_n \times L_1 \times \dots \times L_m 
        \times \Sigma^{*} \rightarrow \omega$ \\
        $R(f, \mathcal{G}) (\overset{\rightarrow}{x}, \overset{\rightarrow}{a}, \varepsilon) = 
        f(\overset{\rightarrow}{x}, \overset{\rightarrow}{a})$ \\
        $R(f, \mathcal{G}) (\overset{\rightarrow}{x}, \overset{\rightarrow}{a}, \alpha a) = 
        \mathcal{G}_a(R(f, \mathcal{G}) (\overset{\rightarrow}{x}, \overset{\rightarrow}{a}, \alpha), 
        \overset{\rightarrow}{x}, \overset{\rightarrow}{a}, \alpha)$
    \end{center}
\end{enumerate}

\end{document} 