\documentclass{article}
\usepackage[utf8]{inputenc}
\usepackage{amsfonts}
\usepackage{amsmath, amssymb}

\title{Lenguajes Formales y Computabilidad \\
        \large Definiciones y Convenciones: Combo 11 }

\author{Nicolás Cagliero}


\begin{document}
\maketitle

Defina:

\begin{enumerate}
    \item $\Psi_{\mathcal{P}}^{n, m, \#}$

    \item "$f$ es $\Sigma$-computable"
    
    \item "$\mathcal{P}$ computa a $f$"
    
    \item $M^{\leq}(P)$
\end{enumerate}

\(\)
\begin{center}
    Respuestas: 
    \(\)
\end{center}

\begin{enumerate}
    \item Dado $\mathcal{P} \in \text{Pro}^{\Sigma}$, definamos para cada
    par $n, m \geq 0$, la función $\Psi_{\mathcal{P}}^{n, m, \#}$ de la siguiente 
    manera:

    \begin{center}
        $D^{n,m,\#}_{\Psi_{\mathcal{P}}} = \left\{ (\vec{x}, \vec{\alpha}) \in \omega^n \times \Sigma^{*m} : \mathcal{P} \text{ termina, partiendo del estado } \parallel x_1, \ldots, x_n, \alpha_1, \ldots, \alpha_m \parallel \right\}$
    \end{center}
    
    \begin{center}
        $\Psi^{n,m,\#}_{\mathcal{P}}(\vec{x}, \vec{\alpha}) = \text{valor de N1 en el estado obtenido cuando } \mathcal{P} \text{ termina, partiendo de } \parallel x_1, \ldots, x_n, \alpha_1, \ldots, \alpha_m \parallel$
    \end{center}

    \item Una función $f : S \subseteq \omega^n \times \Sigma^{*m}
    \rightarrow s$ con $s \in \{\omega, \Sigma^{*}\}$ es llamada $\Sigma-computable$ si hay un programa
    $\mathcal{P}$ de $\mathcal{S}^{\Sigma}$ que la computa

    \item Diremos que $\mathcal{P}$ computa a $f : S \subseteq \omega^n \times \Sigma^{*m}
    \rightarrow s$ con $s \in \{\omega, \Sigma^{*}\}$ si 
    $f = \Psi_{\mathcal{P}}^{n, m, \#}$ si $s = \#$ ó 
    $f = \Psi_{\mathcal{P}}^{n, m, *}$ si $s = *$

    \item Sea $\Sigma$ un alfabeto no vacío, sea $\leq$ un orden total sobre $\Sigma$
     y sea $P : D_P 
    \subseteq \omega^{n} \times \Sigma^{*m} \times \Sigma^{*} \rightarrow \omega$ un predicado.
    Dado $(\overset{\rightarrow}{x}, \overset{\rightarrow}{\alpha})
    \in \omega^n \times \Sigma^{*m}$, cuando exista al menos un $\alpha \in \Sigma^{*}$ tal que
    $P(\overset{\rightarrow}{x}, \overset{\rightarrow}{\alpha}, \alpha) = 1$, 
    usaremos $min_\alpha^{\leq} P(\overset{\rightarrow}{x}, \overset{\rightarrow}{\alpha}, \alpha)$ para
    denotar al menor $\alpha \in \Sigma^{*}$
    tal que $P(\overset{\rightarrow}{x}, \overset{\rightarrow}{\alpha}, \alpha) = 1$. Definimos:
    \begin{center}
        $M^{\leq}(P) = \lambda\overset{\rightarrow}{x} \overset{\rightarrow}{\alpha}
        [min_\alpha^{\leq} P(\overset{\rightarrow}{x}, \overset{\rightarrow}{\alpha}, \alpha)]$
    \end{center}

\end{enumerate}

\end{document} 