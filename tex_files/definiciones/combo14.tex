\documentclass{article}
\usepackage[utf8]{inputenc}
\usepackage{amsfonts}
\usepackage{amsmath, amssymb}

\title{Lenguajes Formales y Computabilidad \\
        \large Definiciones y Convenciones: Combo 14 }

\author{Nicolás Cagliero}


\begin{document}
\maketitle

Explique en forma detallada la notación lambda

\(\)
\begin{center}
    Respuesta: 
    \(\)
\end{center}

Dado un alfabeto finito $\Sigma$, las expresiones que usamos en notación
lambda deben cumplir lo siguiente:
\begin{itemize}
        \item Solo involucran variables numéricas $(x, y, z, t, k, x_1, x_2, y_1, etc)$
        que se valuarán con números en $\omega$, y variables alfabéticas
        $(\alpha, \beta, \gamma, \alpha_1, \alpha_2, \beta_1, etc)$ que se 
        valuarán con palabras de $\Sigma^{*}$
        \item Para ciertas valuaciones de las variables, la expresión puede 
        no estar definida. Por ejemplo $Pred(x)$ con $x = 0$
        \item Toda expresión $E$ debe cumplir alguna de las siguientes propiedades:
        \begin{itemize}
                \item[(a)] Los valores que asuma $E$ cuando hayan sido asignados 
                valores de $\omega$ a sus variables numéricas y valores de $\Sigma^{*}$
                a sus variables alfabéticas de manera que $E$ esté definida para esos valores, 
                deberán ser siempre elementos de $\omega$
                \item[(b)] Los valores que asuma $E$ cuando hayan sido asignados 
                valores de $\omega$ a sus variables numéricas y valores de $\Sigma^{*}$
                a sus variables alfabéticas de manera que $E$ esté definida para esos valores, 
                deberán ser siempre elementos de $\Sigma$
        \end{itemize}
        \item Las expresiones pueden ser escritas en lenguaje coloquial castellano.
        \item Las expresiones booleanas toman valores en $\{0, 1\} \subseteq \omega$: $1$ cuando sean verdaderas, 
        $0$ cuando sean falsas.
   
\end{itemize}

        Las expresiones que cumplan estas propiedades serás llamadas\\
$"lambdificables \text{ } respecto \text{ } a \text{ } \Sigma"$\\
\\

        Ahora, sea $E$ una expresión lambdificables respecto a $\Sigma$.
        Sea $x_1, \dots, x_n, \alpha_1, \dots, \alpha_m$, una lista de variables todas
        distintas tal que las variables numéricas que ocurran en $E$ están contenidas 
        en la lista $x_1, \dots, x_n$ y las alfabéticas que ocurran en $E$ 
        están contenidas en $\alpha_1, \dots, \alpha_m$. Entonces:

        \begin{center}
                $\lambda x_1, \dots, x_n, \alpha_1, \dots, \alpha_m [E]$
        \end{center}

        denotará la función definida por:

        \begin{itemize}
                \item el dominio de $\lambda x_1, \dots, x_n, \alpha_1, \dots, \alpha_m [E]$
                es el conjunto de las $n + m$-uplas 
                $(k_1, \dots, k_n, \beta_1, \dots, \beta_m) \in \omega^{n} \times \Sigma^{*m}$
                tales que $E$ esté definida cuando le asignemos a cada $x_i$ el valor $k_i$ 
                y a cada $\alpha_i$ el valor $\beta_i$
                \item $\lambda x_1, \dots, x_n, \alpha_1, \dots, \alpha_m [E]
                (k_1, \dots, k_n, \beta_1, \dots, \beta_m) = $ valor que ocurre en $E$ cuando 
                asignamos a cada $x_i$ el valor $k_i$ 
                y a cada $\alpha_i$ el valor $\beta_i$
        \end{itemize}
\end{document} 