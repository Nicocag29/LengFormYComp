\documentclass{article}
\usepackage[utf8]{inputenc}
\usepackage{amsfonts}
\usepackage{amsmath, amssymb}
\usepackage{mathrsfs}
\usepackage{enumitem}

\title{Lenguajes Formales y Computabilidad \\
        \large Definiciones y Convenciones: Combo 16 }

\author{Nicolás Cagliero}


\begin{document}
\maketitle

Dado un predicado $P : D_P \subseteq \omega^n \times \Sigma^{*m} \rightarrow \omega$,
describa qué tipo de objeto es y qué propiedades debe tener el macro:
\begin{center}
    $[IF$ $P(V1, W1)$ $GOTO$ $A1]$
\end{center}

\(\)
\begin{center}
    Respuesta: 
    \(\)
\end{center}

Es una palabra que denota el macro $M$ que cumple las siguientes propiedades:
\begin{enumerate}
    \item Las variables oficiales de $M$ son V1, W1
    \item A1 es el único label oficial de $M$
    \item Si reemplazamos:
    \begin{itemize}
        \item las variables oficiales de $M$ por variables concretas $N \bar{k}, P \bar{j}$
        
        \item el label oficial A1 por un label concreto $L \bar{t}$
        
        \item las variables auxiliares de $M$ por variables concretas (distintas de a dos) y distintas de $N \bar{k}, P \bar{j}$
        
        \item los labels auxiliares de $M$ por labels concretos (distintos de a dos) y ninguno igual a $L \bar{t}$
    \end{itemize}
    entonces la palabra así obtenida es un programa $\mathcal{E}$ de $\mathcal{S}^{\Sigma}$ (salvo por la ley de los GOTO respecto de $L \bar{t}$) que denotaremos en general con

    \[
    [\text{IF } P(N \bar{k}, P \bar{j}) \text{ GOTO } L \bar{t}]
    \]

    el cual debe tener la siguiente propiedad:

    \begin{description}
        \item[(E)] Si hacemos correr $\mathcal{E}$ partiendo de un estado $e$ que le asigne a las variables $N \bar{k}, P \bar{j}$ valores $x_1, \alpha_1$, entonces independientemente de los valores que les asigne $e$ al resto de las variables (incluidas las que fueron a reemplazar a las variables auxiliares de $M$) se dará que:
        \begin{itemize}
            \item[(i)] si $(x_1, \alpha_1) \notin D_P$, entonces $\mathcal{E}$ no se detiene
            \item[(ii)] si $(x_1, \alpha_1) \in D_P$ y $P(x_1, \alpha_1) = 1$, entonces luego de una cantidad finita de pasos, $\mathcal{E}$ direcciona al label $L \bar{t}$ quedando en un estado $e'$ el cual solo puede diferir de $e$ en los valores que le asigna a las variables que fueron a reemplazar a las variables auxiliares de $M$. 
            \item[(iii)] si $(x_1, \alpha_1) \in D_P$ y $P(x_1, \alpha_1) = 0$, entonces luego de una cantidad finita de pasos, el programa se detiene quedando en un estado $e'$ el cual solo puede diferir de $e$ en los valores que le asigna a las variables que fueron a reemplazar a las variables auxiliares de $M$. 
        \end{itemize}
    \end{description}
\end{enumerate}
\end{document} 