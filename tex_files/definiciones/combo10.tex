\documentclass{article}
\usepackage[utf8]{inputenc}
\usepackage{amsfonts}
\usepackage{amsmath, amssymb}
\title{Lenguajes Formales y Computabilidad \\
        \large Definiciones y Convenciones: Combo 10 }

\author{Nicolás Cagliero}


\begin{document}
\maketitle

Defina relativo al lenguaje $S^{\Sigma}$
\begin{enumerate}
    \item "estado"

    \item "descripción instantánea"
    
    \item $S_{\mathcal{P}}$
    
    \item "estado obtenido luego de $t$ pasos, partiendo del estado 
    $(\overset{\rightarrow}{s}, \overset{\rightarrow}{\sigma})$"
    
    \item "$\mathcal{P}$ se detiene (luego de $t$ pasos), partiendo del estado
    $(\overset{\rightarrow}{s}, \overset{\rightarrow}{\sigma})$"
\end{enumerate}

\(\)
\begin{center}
    Respuestas: 
    \(\)
\end{center}

\begin{enumerate}
    \item Un estado es un par $(\overset{\rightarrow}{s}, \overset{\rightarrow}{\sigma}) = 
    ((s_1, s_2, \dots), (\sigma_1, \sigma_2, \dots))
    \in \omega^{[\mathbf{N}]} \times \Sigma^{*[\mathbf{N}]}$

    \item Una descripción instantánea es una terna 
    $(i, \overset{\rightarrow}{s}, \overset{\rightarrow}{\sigma})$ tal que 
    $(\overset{\rightarrow}{s}, \overset{\rightarrow}{\sigma})$ es un estado e $i \in \omega$

    \item $S_{\mathcal{P}} : \omega \times \omega^{[\mathbf{N}]} \times \Sigma^{*[\mathbf{N}]}
    \rightarrow \omega \times \omega^{[\mathbf{N}]} \times \Sigma^{*[\mathbf{N}]}$

    \begin{center}
        $S_{\mathcal{P}} (i, \overset{\rightarrow}{s}, \overset{\rightarrow}{\sigma}) = $
        descripción instantánea que resulta luego de realizar $I_i^{\mathcal{P}}$,
        estando en estado $(\overset{\rightarrow}{s}, \overset{\rightarrow}{\sigma})$
    \end{center}

    \item Si
                \[
                \overbrace{S_{\mathcal{P}}(\dots S_{\mathcal{P}}(S_{\mathcal{P}}(1, \vec{s}, \vec{\sigma})) \dots)}^{t \text{ veces}} = (j, \vec{u}, \vec{\eta})
                \]
                diremos que \( (\vec{u}, \vec{\eta}) \) es el \textit{estado obtenido luego de \( t \) pasos, partiendo del estado \( (\vec{s}, \vec{\sigma}) \)}.

    \item Cuando la primer coordenada de 
                \[
                \overbrace{S_{\mathcal{P}}(\dots S_{\mathcal{P}}(S_{\mathcal{P}}(1, \vec{s}, \vec{\sigma})) \dots)}^{t \text{ veces}}
                \]
                sea igual a $n(\mathcal{P}) + 1$, diremos que $\mathcal{P}$ 
                \textit{se detiene (luego de $t$ pasos), partiendo
                desde el estado} $(\overset{\rightarrow}{s}, \overset{\rightarrow}{\sigma})$
\end{enumerate}

\end{document} 