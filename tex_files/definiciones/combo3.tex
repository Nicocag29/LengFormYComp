\documentclass{article}
\usepackage[utf8]{inputenc}
\usepackage{amsmath, amssymb}
\title{Lenguajes Formales y Computabilidad \\
        \large Definiciones y Convenciones: Combo 3 }

\author{Nicolás Cagliero}


\begin{document}
\maketitle

\begin{enumerate}
    \item Defina cuando un conjunto $S \subseteq \omega^n \times \Sigma^{*m}$ 
    es llamado $\Sigma$-recursivamente enumerable (no hace falta que defina "función $\Sigma$-recursiva")

    \item Defina $s^{\leq}$
    
    \item Defina $*^{\leq}$
    
    \item Defina $\#^{\leq}$
\end{enumerate}
\(\)
\begin{center}
    Respuestas: 
    \(\)
\end{center}

\begin{enumerate}
    \item Diremos que un conjunto  $S \subseteq \omega^n \times \Sigma^{*m}$ 
    es llamado $\Sigma$-recursivamente enumerable cuando sea vacío o haya
    una función $F : \omega \rightarrow \omega^{n} \times \Sigma^{*m}$
    tal que $I_F = S$ y $F_i$ sea $\Sigma$-recursiva, para cada
    $i \in \{1, \dots, n+m\}$

    \item $s^{\leq} : \Sigma^{*} \rightarrow \Sigma^{*}$
    \begin{align*}
        &s^{\leq}((a_n)^{m}) = (a_1)^{m+1}, \text{ para cada } m \geq 0 \\
        &s^{\leq}(\alpha a_i(a_n)^{m}) = \alpha a_{i+1}(a_1)^{m}
    \end{align*}

    \item $*^{\leq} : \omega \rightarrow \Sigma^{*}$
    \begin{align*}
        &*^{\leq}(0) = \varepsilon \\
        &*^{\leq}(i + 1) = s^{\leq}(*^{\leq}(i))
    \end{align*}

    \item $\#^{\leq} : \Sigma^{*} \rightarrow \omega$
    \begin{align*}
        &\#^{\leq}(\varepsilon) = 0 \\
        &\#^{\leq}(a_{i_k} \dots a_{i_0}) = i_k n^k + \dots + i_0 n^0 
        \text{ para } i_0, i_1, \dots, i_k \in \{1, \dots, n\}
    \end{align*}
\end{enumerate}

\end{document} 