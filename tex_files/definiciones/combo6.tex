\documentclass{article}
\usepackage[utf8]{inputenc}
\usepackage{amsmath, amssymb}
\title{Lenguajes Formales y Computabilidad \\
        \large Definiciones y Convenciones: Combo 6 }

\author{Nicolás Cagliero}


\begin{document}
\maketitle

Defina cuando un conjunto $S \subseteq \omega^n \times \Sigma^{*m}$
es llamado $\Sigma$-efectivamente enumerable y defina: "el procedimiento
efectivo $\mathbb{P}$ enumera a $S$"

\(\)
\begin{center}
    Respuesta: 
    \(\)
\end{center}

Un conjunto $S \subseteq \omega^n \times \Sigma^{*m}$
es llamado $\Sigma$-efectivamente enumerable cuando sea vacío
o haya una función $F : \omega \rightarrow \omega^n \times \Sigma^{*m}$ 
tal que $I_f = S$ y $F_{(i)}$ sea $\Sigma$-efectivamente computable
para cada $i \in \{1, \dots, n + m\}$\\

Diremos que "el procedimiento efectivo $\mathbb{P}$ enumera a $S$"
cuando un procedimiento efectivo $\mathbb{P}$ cumpla:
\begin{align*}
        &\text{(1) El conjunto de datos de entrada de } \mathbb{P}
        \text{ es } \omega\\
        &\text{(2) } \mathbb{P} \text{ se detiene para cada } x \in S\\
        &\text{(3) El conjunto de datos de salida de } \mathbb{P}
        \text{ es igual a } S. \text{ Es decir, siempre que } \mathbb{P}
        \text{ se detiene, }\\
        &\text{da como salida un elemento de } S
        \text{, y para cada elemento} 
        (\overset{\rightarrow}{x}, \overset{\rightarrow}{\alpha})
        \in S \text{, hay un }\\
        &x \in \omega \text{ tal que } \mathbb{P}
        \text{ da como salida a } 
        (\overset{\rightarrow}{x}, \overset{\rightarrow}{\alpha}) 
        \text{ cuando lo corremos con } x \text{ como dato de entrada }
\end{align*}

\end{document} 