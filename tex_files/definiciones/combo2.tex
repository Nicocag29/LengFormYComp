\documentclass{article}
\usepackage[utf8]{inputenc}
\usepackage{amsmath, amssymb}

\title{Lenguajes Formales y Computabilidad \\
        \large Definiciones y Convenciones: Combo 2 }

\author{Nicolás Cagliero}


\begin{document}
\maketitle

Defina:
\begin{enumerate}
    \item $d \overset{n}{\vdash} d'$ y $d \overset{*}{\vdash} d'$
            (no hace falta que defina $\vdash$)

    \item $L(M)$
    
    \item "$f$ es una función de tipo $(n, m, s)$"
    
    \item $(x)$
    
    \item $(x)_i$
\end{enumerate}
\(\)
\begin{center}
    Respuestas: 
    \(\)
\end{center}

\begin{enumerate}
        \item Para $d, d' \in Des$ y $n \geq 0$, escribiremos $d \overset{n}{\vdash} d'$
        si existen $d_1, \dots, d_{n+1} \in Des$ tales que 
        \begin{align*}
                d &= d_1 \\
                d' &= d_{n+1} \\
                d_i &\vdash d_{i+1}, \quad \text{para } i = 1, \ldots, n
        \end{align*}
        Por último: $d \overset{*}{\vdash} d'$ sii $(\exists n \in \omega)$ $d \overset{n}{\vdash} d'$

        \item Diremos que una palabra $\alpha \in \Sigma^{*} \text{ } es \text{ } 
        aceptada \text{ } por \text{ } M \text{ } por \text{ } alcance \text{ }
        de \text{ } estado \text{ } final \text{ } cuando$
        \begin{center}
                $\left\lceil q_0 B \alpha \right\rceil \vdash^* d, 
                \text{ con } d \text{ tal que } St(d) \in F.$
        \end{center}
        El $lenguaje \text{ } aceptado \text{ } por \text{ } M \text{ } por alcance
        \text{ } de \text{ } estado \text{ } final$ se define de la siguiente manera
        \begin{center}
                $L(M) = \{ \alpha \in \Sigma^{*}: \text{es aceptada por } M \text{ por 
                alcance de estado final}\}$
        \end{center}

        \item $f : D_f \subseteq \omega^n \times \Sigma^{*m}$ y:
        \begin{align*}
                s &= \# \text{ y } I_f \subseteq \omega \\
                \text{ ó } \\
                s &= * \text{ y } I_f \subseteq \Sigma^{*}
        \end{align*}

        \item Dado $x \in \mathbf{N}$, usaremos $(x)$ para denotar la única infinitupla
        $(s_1, s_2, \dots) \in \omega^{\mathbf{N}}$ tal que
        \begin{center}
                $x = \langle s_1, s_2, \dots \rangle = \displaystyle \prod_{i=1}^{\infty} pr(i)^{s_i}$
        \end{center}

        \item Para $i \in \mathbf{N}$, usaremos $(x)_i$ para denotar a $s_i$ de la infinitupla
        $(x)$. Es decir que 
        \begin{enumerate}
                \item $(x) = ((x)_1, (x)_2, \dots)$
                \item $(x)_i$ es el exponente de $pr(i)$ en la (única posible) factorización de 
                $x$ como producto de primos

        \end{enumerate}
\end{enumerate}

\end{document} 