\documentclass{article}
\usepackage[utf8]{inputenc}
\usepackage{amsmath, amssymb}
\title{Lenguajes Formales y Computabilidad \\
        \large Definiciones y Convenciones: Combo 8 }

\author{Nicolás Cagliero}


\begin{document}
\maketitle

Defina:

\begin{enumerate}
    \item $M(P)$

    \item $Lt$
    
    \item Conjunto rectangular
    
    \item "$S$ es un conjunto de tipo $(n, m)$"
\end{enumerate}

\(\)
\begin{center}
    Respuestas: 
    \(\)
\end{center}

\begin{enumerate}
    \item Sea $\Sigma$ un alfabeto finito y sea $P : D_P 
    \subseteq \omega \times \omega^{n} \times \Sigma^{*m} \rightarrow \omega$ un predicado.
    Dado $(\overset{\rightarrow}{x}, \overset{\rightarrow}{\alpha})
    \in \omega^n \times \Sigma^{*m}$, cuando exista al menos un $t \in \omega$ tal que
    $P(t, \overset{\rightarrow}{x}, \overset{\rightarrow}{\alpha}) = 1$, 
    usaremos $min_t P(t, \overset{\rightarrow}{x}, \overset{\rightarrow}{\alpha})$ para
    denotar al menor de tales $t's$. Definimos:
    \begin{center}
        $M(P) = \lambda\overset{\rightarrow}{x} \overset{\rightarrow}{\alpha}
        [min_t P(t, \overset{\rightarrow}{x}, \overset{\rightarrow}{\alpha})]$
    \end{center}

    \item $Lt : \mathbf{N} \rightarrow \omega$
    \begin{center}
        \[
            Lt(x) =
            \begin{cases}
            \max_i (x)_i \neq 0 & \text{si } x \neq 1 \\
            0 & \text{si } x = 1
            \end{cases}
        \]
    \end{center}

    \item Sea $\Sigma$ un alfabeto finito. Un conjunto $\Sigma$-mixto $S$ es llamado
    $rectangular$ si es de la forma 
    \begin{center}
        $S_1 \times \dots \times S_n \times L_1 \times \dots \times L_m$
    \end{center}
    con cada $S_i \subseteq \omega$ y cada $L_i \subseteq \Sigma^{*}$

    \item $n, m \in \omega$ son tales que 
    $S \subseteq \omega^n \times \Sigma^{*m}$
\end{enumerate}

\end{document} 