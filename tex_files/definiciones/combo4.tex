\documentclass{article}
\usepackage[utf8]{inputenc}
\usepackage{amsmath, amssymb}
\title{Lenguajes Formales y Computabilidad \\
        \large Definiciones y Convenciones: Combo 4 }

\author{Nicolás Cagliero}


\begin{document}
\maketitle

Defina cuando una función $f : D_f \subseteq \omega^n \times \Sigma^{*m} 
\rightarrow \omega$
es llamada $\Sigma$-efectivamente computable y defina: "el procedimiento
efectivo $\mathbb{P}$ computa a la función $f$"

\(\)
\begin{center}
    Respuesta: 
    \(\)
\end{center}

Una función $\Sigma \text{-mixta } f : D_f \subseteq \omega^n \times \Sigma^{*m} 
\rightarrow \omega$ es llamada $\Sigma-efectivamente \text{ } computable$
cuando hay un procedimiento efectivo $\mathbb{P}$ que la computa. Esto es:
\begin{align*}
        &\text{(1) El conjunto de datos de entrada de } \mathbb{P}
        \text{ es } \omega^n \times \Sigma^{*m}\\
        &\text{(2) El conjunto de datos de salida está contenido en } \Sigma^{*}\\
        &\text{(3) Si } (\overset{\rightarrow}{x}, \overset{\rightarrow}{\alpha})
        \in D_f \text{, entonces } \mathbb{P} \text{ se detiene partiendo de }
        (\overset{\rightarrow}{x}, \overset{\rightarrow}{\alpha}) \text{,
         dando como dato de salida } f(\overset{\rightarrow}{x}, \overset{\rightarrow}{\alpha})\\
        &\text{(4) Si } (\overset{\rightarrow}{x}, \overset{\rightarrow}{\alpha})
        \in (\omega^n \times \Sigma^{*m}) - D_f \text{, entonces
         } \mathbb{P} \text{ no se detiene partiendo de } 
        (\overset{\rightarrow}{x}, \overset{\rightarrow}{\alpha})
\end{align*}


\end{document} 