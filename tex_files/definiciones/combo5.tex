\documentclass{article}
\usepackage[utf8]{inputenc}
\usepackage{amsmath, amssymb}
\title{Lenguajes Formales y Computabilidad \\
        \large Definiciones y Convenciones: Combo 5 }

\author{Nicolás Cagliero}


\begin{document}
\maketitle

Defina cuando un conjunto $S \subseteq \omega^n \times \Sigma^{*m}$
es llamado $\Sigma$-efectivamente computable y defina: "el procedimiento
efectivo $\mathbb{P}$ decide la pertenencia a $S$"

\(\)
\begin{center}
    Respuesta: 
    \(\)
\end{center}

Un conjunto $S \subseteq \omega^n \times \Sigma^{*m}$
es llamado $\Sigma$-efectivamente computable cuando la función
$\chi_{S}^{\omega^n \times \Sigma^{*m}}$ sea $\Sigma$-efectivamente
computable. O sea $S$ es $\Sigma$-efectivamente computable
sii hay un procedimiento efectivo $\mathbb{P}$, el cual 
computa a $\chi_{S}^{\omega^n \times \Sigma^{*m}}$, es decir
\begin{align*}
        &\text{- El conjunto de datos de entrada de }
        \mathbb{P} \text{ es } \omega^n \times \Sigma^{*m} 
        \text{, siempre termina y da como }\\
        &\text{dato de salida un elemento de } \{0, 1\}.\\
        &\text{- Dado }(\overset{\rightarrow}{x}, \overset{\rightarrow}{\alpha})
        \in \omega^n \times \Sigma^{*m}, \text{ } \mathbb{P} 
        \text{ da como salida al número } 1 \text{ si }
        (\overset{\rightarrow}{x}, \overset{\rightarrow}{\alpha}) \in
        S \text{ y al número }\\
        &0 \text{ si } (\overset{\rightarrow}{x}, \overset{\rightarrow}{\alpha}) \notin S. 
\end{align*}

\end{document} 