\documentclass{article}
\usepackage[utf8]{inputenc}
\usepackage{amsmath, amssymb}
\usepackage{mathrsfs}
\usepackage{enumitem}

\title{Lenguajes Formales y Computabilidad \\
        \large Definiciones y Convenciones: Combo 17 }

\author{Nicolás Cagliero}


\begin{document}
\maketitle

Defina el concepto de función y desarrolle las tres Convenciones
Notacionales asociadas a dicho concepto (Guía 1)

\(\)
\begin{center}
    Respuesta: 
    \(\)
\end{center}

Una $funcion$ es un conjunto $f$ de pares ordenados con la siguiente propiedad:
\begin{center}
        Si $(x, y) \in f$ y $(x, z) \in f$, entonces $y = z$
\end{center}
\(\)

\begin{itemize}
        \item[\textbf{Convención Notacional 1:}] dado $x \in D_f$, usaremos 
        $f(x)$ para denotar al único $y \in I_f$ tal que $(x, y) \in f$
        \item[\textbf{Convención Notacional 2:}] escribiremos $f : S \subseteq A \rightarrow B$ 
        para expresar que $f$ es una función tal que $D_f = S \subseteq A$ y $I_f \subseteq B$.
        También escribiremos $f : A \rightarrow B$ para expresar para expresar que $f$ es una 
        función tal que $D_f = A$ y $I_f \subseteq B$. En tal contexto llamaremos a 
        $B$ \textit{conjunto de llegada}. Por supuesto $B$ no está determinado por $f$ ya que 
        solo debe cumplir $I_f \subseteq B$. Es decir que cualquier conjunto $B$ que contenga a 
        $I_f$ puede ser considerado conjunto de llegada de $f$.
        \item[\textbf{Convención Notacional 3:}] muchas veces para definir una función $f$, lo haremos 
        dando su dominio y su regla de asignación, es decir especificaremos en forma precisa qué 
        conjunto es el dominio de $f$ y además especificaremos en forma precisa quién es $f(x)$ para cada 
        $x$ de dicho dominio. Obviamente esto determina por completo a la función $f$ ya que siempre se da 
        que $f = \{(x, f(x)) : x \in D_f\}$.
\end{itemize}

\end{document} 