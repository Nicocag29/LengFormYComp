\documentclass{article}
\usepackage[utf8]{inputenc}
\usepackage{amsfonts}
\usepackage{amsmath, amssymb}
\usepackage{mathrsfs}
\usepackage{enumitem}


\title{Lenguajes Formales y Computabilidad \\
        \large Definiciones y Convenciones: Combo 15 }

\author{Nicolás Cagliero}


\begin{document}
\maketitle

Dada una función $f : D_f \subseteq \omega^n \times \Sigma^{*m} \rightarrow \omega$,
describa qué tipo de objeto es y qué propiedades debe tener el macro:
\begin{center}
    $[V2 \leftarrow f(V1, W1)]$
\end{center}

\(\)
\begin{center}
    Respuestas: 
    \(\)
\end{center}

Ese macro es una palabra que cumple:
\begin{enumerate}
    \item Sus variables oficiales son $V1, V2, W1$
    \item No tiene labels oficiales
    \item Si reemplazamos las variables oficiales por variables concretas
    $N \bar{k_1}, N \bar{k_2}, P \bar{j_1}$, las variable auxiliares por 
    variables concretas distintas dos a dos y no en 
    $N \bar{k_1}, N \bar{k_2}, P \bar{j_1}$ y los lables auxiliares 
    por labels concretos distintos dos a dos, entonces la palabra obtenida 
    es un programa $\mathcal{P} \in \text{Pro}^{\Sigma}$ que denotaremos
    con 
    \begin{center}
        $[N \bar{k_2} \leftarrow f( N \bar{k_1}, P \bar{j_1})]$
    \end{center}
    el cual debe tener la siguiente propiedad:\\
    Si hacemos correr $\mathcal{P}$ partiendo de un estado $e$ que le asigne a las variables 
$N \bar{k_1}, P \bar{j_1}$ 
valores $x_1, \alpha_1$, entonces, independientemente de los valores que les asigne $e$ al resto de las variables 
se dará que

    \begin{itemize}
        \item si $(x_1, \alpha_1) \notin D_f$, entonces $\mathcal{P}$ no se detiene
        \item si $(x_1, \alpha_1) \in D_f$, entonces $\mathcal{P}$ se detiene y llega a un estado $e'$ el cual cumple:
        \begin{itemize}
            \item $e'$ le asigna a $N \bar{k_2}$ el valor $f(x_1, \alpha_1)$
            \item $e'$ solo puede diferir de $e$ en los valores que le asigna a $N \bar{k_2}$ o 
            las variables auxiliares.
        \end{itemize}
    \end{itemize}

\end{enumerate}

\end{document} 