\documentclass{article}
\usepackage[utf8]{inputenc}
\usepackage{amsfonts}
\usepackage{amsmath, amssymb}

\title{Lenguajes Formales y Computabilidad \\
        \large Definiciones y Convenciones: Combo 13 }

\author{Nicolás Cagliero}


\begin{document}
\maketitle

Defina:

\begin{enumerate}
    \item $i^{n, m}$
    \item $E_{\#}^{n, m}$
    \item $E_{*}^{n, m}$
    \item $E_{\#j}^{n, m}$
    \item $E_{*j}^{n, m}$
    \item $Halt^{n, m}$
    \item $T^{n, m}$
    \item $AutoHalt^{\Sigma}$
    \item Los conjuntos $A$ y $N$
\end{enumerate}

\(\)
\begin{center}
    Respuestas: 
    \(\)
\end{center}

\begin{enumerate}
    \item Sean $n, m \in \omega$ fijos. Definamos
    \[
        i^{n,m} : \omega \times \omega^n \times \Sigma^{*m} \times \text{Pro}^\Sigma \to \omega
        \]
        \[
        E_{\#}^{n,m} : \omega \times \omega^n \times \Sigma^{*m} \times \text{Pro}^\Sigma \to \omega^{[\mathbf{N}]}
        \]
        \[
        E_{*}^{n,m} : \omega \times \omega^n \times \Sigma^{*m} \times \text{Pro}^\Sigma \to \Sigma^{*[\mathbf{N}]}
    \]

de la siguiente manera

\begin{align*}
&(i^{n,m}(0, \vec{x}, \vec{\alpha}, \mathcal{P}), E_{\#}^{n,m}(0, \vec{x}, \vec{\alpha}, \mathcal{P}), E_{*}^{n,m}(0, \vec{x}, \vec{\alpha}, \mathcal{P})) \\
&= (1, (x_1, ..., x_n, 0, ...), (\alpha_1, ..., \alpha_m, \varepsilon, ...)) \\
&(i^{n,m}(t+1, \vec{x}, \vec{\alpha}, \mathcal{P}), E_{\#}^{n,m}(t+1, \vec{x}, \vec{\alpha}, \mathcal{P}), E_{*}^{n,m}(t+1, \vec{x}, \vec{\alpha}, \mathcal{P})) \\
&= S_\mathcal{P}(i^{n,m}(t, \vec{x}, \vec{\alpha}, \mathcal{P}), E_{\#}^{n,m}(t, \vec{x}, \vec{\alpha}, \mathcal{P}), E_{*}^{n,m}(t, \vec{x}, \vec{\alpha}, \mathcal{P}))
\end{align*}

\item Definamos para cada $j \in \mathbf{N}$, funciones
\begin{center}
    $E_{\#j}^{n, m} : \omega \times \omega^n \times \Sigma^{*m} \times \text{Pro}^{\Sigma} \rightarrow \omega$\\
    $E_{*j}^{n, m} : \omega \times \omega^n \times \Sigma^{*m} \times \text{Pro}^{\Sigma} \rightarrow \Sigma^{*}$\\
\end{center}
de la siguiente manera:
\begin{center}
    $E_{\#j}^{n, m}(t, \vec{x}, \vec{\alpha}, \mathcal{P}) = j-\text{ésima coordenada de } E_{\#}^{n, m}(t, \vec{x}, \vec{\alpha}, \mathcal{P})$\\
    $E_{*sj}^{n, m}(t, \vec{x}, \vec{\alpha}, \mathcal{P}) = j-\text{ésima coordenada de } E_{*}^{n, m}(t, \vec{x}, \vec{\alpha}, \mathcal{P})$\\
\end{center}

\item Dados $n, m \in \omega$, definamos:
\begin{center}
    $Halt^{n, m} = \lambda t \vec{x} \vec{\alpha} \mathcal{P} [i^{n, m}(t, \vec{x}, \vec{\alpha}, \mathcal{P}) = n(\mathcal{P}) + 1]$
\end{center}

\item Dados $n, m \in \omega$, $T^{n, m} = M(Halt^{n, m})$
\item Cuando $\Sigma \supseteq \Sigma_p$, definimos:
\begin{center}
    $AutoHalt^{\Sigma} = \lambda \mathcal{P}[(\exists t \in \omega) \text{ } Halt^{0,1}(t, \mathcal{P}, \mathcal{P})]$
\end{center}

\item Cuando $\Sigma \supseteq \Sigma_p$, definimos:
\begin{center}
    $A = \{\mathcal{P} \in \text{Pro}^{\Sigma} : AutoHalt^{\Sigma}(\mathcal{P}) = 1 \}$\\
    $N = \{\mathcal{P} \in \text{Pro}^{\Sigma} : AutoHalt^{\Sigma}(\mathcal{P}) = 0 \}$\\
\end{center}

\end{enumerate}

\end{document} 