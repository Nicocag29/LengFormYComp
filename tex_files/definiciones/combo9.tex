\documentclass{article}
\usepackage[utf8]{inputenc}
\usepackage{amsfonts}
\usepackage{amsmath, amssymb}
\title{Lenguajes Formales y Computabilidad \\
        \large Definiciones y Convenciones: Combo 9 }

\author{Nicolás Cagliero}


\begin{document}
\maketitle

Defina:

\begin{enumerate}
    \item "$I$ es una instrucción de $\mathcal{S}^{\Sigma}$"
    
    \item "$\mathcal{P}$ es un programa de $\mathcal{S}^{\Sigma}$"
    
    \item $I_i^{\mathcal{P}}$
    
    \item $n(\mathcal{P})$
    
    \item $Bas$
\end{enumerate}

\(\)
\begin{center}
    Respuestas: 
    \(\)
\end{center}

\begin{enumerate}
    \item $I$ es una instrucción básica de $\mathcal{S}^{\Sigma}$ 
    ó una palabra de la forma $\alpha I'$, donde $\alpha \in
    \{L \bar{n} : n \in \mathbf{N} \}$ e $I'$ es una instrucción
    básica de $\mathcal{S}^{\Sigma}$.

    \item $\mathcal{P}$ es una palabra de la forma
    \begin{center}
        $I_1 I_2 \dots I_n$
    \end{center}
    donde $n \geq 1, I_1 I_2 \dots I_n \in \text{Ins}^{\Sigma}$. Y además
    se cumple la siguiente propiedad: para cada $i \in \{1, \dots, n\}$,
    si $\text{GOTOL}\bar{m}$ es un tramo final de $I_i$, entonces
    existe $j \in \{1, \dots, n\}$ tal que $I_j$ tiene label
    $\text{L}\bar{m}$

    \item $i$-ésima instrucción de la sucesión de instrucciones 
    $I_1, \dots, I_n$ \\tal que $\mathcal{P} = I_1 \dots I_n$ ó $\varepsilon$
    cuando $i = 0$ ó $i > n(\mathcal{P})$

    \item Cantidad de instrucciones de $\mathcal{P}$. Es decir 
    $\mathcal{P} = I_1 \dots I_{n(\mathcal{P})}$.

    \item $Bas : \text{Ins}^{\Sigma} \rightarrow (\Sigma \cup \Sigma_p)^{*}$
    \begin{center}
        
    \[
        Bas(I) =
        \begin{cases}
        J & \text{si } I \text{ es de la forma } L\bar{k}J \text{ con } J \in \text{Ins}^\Sigma \\
        I & \text{caso contrario}
        \end{cases}
    \]
    \end{center}
\end{enumerate}

\end{document} 