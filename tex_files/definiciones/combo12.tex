\documentclass{article}
\usepackage[utf8]{inputenc}
\usepackage{amsfonts}
\usepackage{amsmath, amssymb}

\title{Lenguajes Formales y Computabilidad \\
        \large Definiciones y Convenciones: Combo 12 }

\author{Nicolás Cagliero}


\begin{document}
\maketitle

Defina cuando un conjunto $S \subseteq \omega^n \times \Sigma^{*m}$ es llamado
$\Sigma$-computable, cuando es llamado $\Sigma$-enumerable y defina 
"el programa $\mathcal{P}$ enumera a $S$"

\(\)
\begin{center}
    Respuesta: 
    \(\)
\end{center}

Un conjunto $S \subseteq \omega^n \times \Sigma^{*m}$ es llamado
$\Sigma$-computable cuando la función $\chi_S^{\omega^n \times \Sigma^{*m}}$ sea $\Sigma$-computable.\\
\\

Un conjunto $S \subseteq \omega^n \times \Sigma^{*m}$ es llamado $\Sigma$-enumerable
cuando sea vacío o haya una función $F : \omega \rightarrow \omega^n \times \Sigma^{*m}$
tal que $I_F = S$ y $F_{(i)}$ sea $\Sigma$-computable, para cada $i \in \{1, \dots, n + m\}$\\
\\

Un programa $\mathcal{P} \in \text{Pro}^{\Sigma}$  enumera a $S$ cuando:
\begin{itemize}
  \item[(a)] \textit{Para cada $x \in \omega$, tenemos que $\mathcal{P}$ se detiene partiendo desde el estado $\|x\|$ y llega a un estado de la forma $((x_1, ..., x_n, y_1, ...), (\alpha_1, ..., \alpha_m, \beta_1, ...))$, donde $(x_1, ..., x_n, \alpha_1, ..., \alpha_m) \in S$.}
  
  \item[(b)] \textit{Para cada $(x_1, ... x_n, \alpha_1, ..., \alpha_m) \in S$ hay un $x \in \omega$ tal que $\mathcal{P}$ se detiene partiendo desde el estado $\|x\|$ y llega a un estado de la forma}\\
  \textit{$((x_1, ..., x_n, y_1, ...), (\alpha_1, ..., \alpha_m, \beta_1, ...))$} 
\end{itemize}
\end{document} 