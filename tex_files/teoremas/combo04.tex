\documentclass{article}
\usepackage[utf8]{inputenc}
\usepackage{amsfonts}
\usepackage{amsmath, amssymb}
\usepackage{mathrsfs}
\usepackage{enumitem}
\usepackage{stmaryrd}


\title{Lenguajes Formales y Computabilidad \\
        \large Teoremas: Combo 4 }

\author{Nicolás Cagliero}

\begin{document}
\maketitle

\textbf{Proposición} (Caracterización básica de conjuntos $\Sigma$-enumerables). 
\textit{Sea $S \subseteq \omega^n \times \Sigma^{*m}$ un conjunto no vacío. Entonces son equivalentes:}

\begin{enumerate}
    \item $S$ es $\Sigma$-enumerable.
    \item Hay un programa $\mathcal{P} \in \text{Pro}^{\Sigma}$ tal que:
    \begin{enumerate}
        \item \textit{Para cada $x \in \omega$, tenemos que $\mathcal{P}$ se detiene partiendo desde el estado 
        $\llbracket x \rrbracket$ y llega a un estado de la forma $((x_1, ..., x_n, y_1, ...), (\alpha_1, ..., \alpha_m, \beta_1, ...))$, 
        donde $(x_1, ..., x_n, \alpha_1, ..., \alpha_m) \in S$.}
        
        \item \textit{Para cada $(x_1, ..., x_n, \alpha_1, ..., \alpha_m) \in S$ hay un $x \in \omega$ tal que $\mathcal{P}$ 
        se detiene partiendo desde el estado $\llbracket x \rrbracket$ y llega a un estado de la forma 
        $((x_1, ..., x_n, y_1, ...), (\alpha_1, ..., \alpha_m, \beta_1, ...))$.}
    \end{enumerate}
\end{enumerate}

\medskip

(Hacer el caso $n = 2$ y $m = 1$)
\medskip

\textit{Proof.} (1) $\Rightarrow$ (2). Ya que $S$ es no vacío, por definición existe una función 
\[
F : \omega \to \omega^2 \times \Sigma^*
\]
tal que $I_F = S$ y cada componente $F_{(i)}$ es $\Sigma$-computable, para $i = 1, 2, 3$.

Por el Primer Manantial de Macros, existen los siguientes macros:
\[
\begin{aligned}
&[\text{V2} \leftarrow F_{(1)}(\text{V1})] \\
&[\text{V2} \leftarrow F_{(2)}(\text{V1})] \\
&[\text{W1} \leftarrow F_{(3)}(\text{V1})]
\end{aligned}
\]

Sea $\mathcal{P}$ el siguiente programa:
\[
\begin{aligned}
&[\text{P}1 \leftarrow F_{(3)}(\text{N1})] \\
&[\text{N2} \leftarrow F_{(2)}(\text{N1})] \\
&[\text{N1} \leftarrow F_{(1)}(\text{N1})]
\end{aligned}
\]

donde se supone que las expansiones de los macros usan variables auxiliares que no aparecen en la lista $\text{N1}, \text{N2}, \text{P1}$ y tampoco se repiten labels auxiliares.\\
Ver que se cumplan las condiciones de 2 es fácil: nuestro programa copia el comportamiento de cada componente de $F$ que ya sabemos que enumera a $S$, por lo tanto, para cada $x\in \omega$ nuestro programa termina y llega a un estado con la forma esperada y además, sabemos que para cada elemento de $S$ existe algún $x \in \omega$ que lo enumera, por lo tanto, nuestro programa se detendrá en ese $x$ llegando a un estado de la forma esperada. 
\medskip

(2) $\Rightarrow$ (1). Supongamos que $\mathcal{P} \in \text{Pro}^\Sigma$ cumple (a) y (b) de (2). Sea:

\[
\begin{aligned}
&\mathcal{P}_1 = \mathcal{P} \text{N1} \leftarrow \text{N1} \\
&\mathcal{P}_2 = \mathcal{P} \text{N1} \leftarrow \text{N2} \\
&\mathcal{P}_3 = \mathcal{P}\text{P1} \leftarrow \text{P2}
\end{aligned}
\]

Definimos entonces:
\[
\begin{aligned}
&F_1 = \Psi^{1,0,\#}_{\mathcal{P}_1} \\
&F_2 = \Psi^{1,0,\#}_{\mathcal{P}_2} \\
&F_3 = \Psi^{1,0,*}_{\mathcal{P}_3}
\end{aligned}
\]

Nótese que cada $F_i$ es $\Sigma$-computable y tiene dominio $\omega$. Sea $F = [F_1, F_2, F_3]$. Por definición, $D_F = \omega$ y como $F_{(i)} = F_i$ para $i = 1, 2, 3$, tenemos que cada componente de $F$ es $\Sigma$-computable.

Necesito verificar que $I_F = S$. \\
Aclaración 1: 
\begin{center}
    $\mathcal{P}_1$ deja en N1 el valor que $\mathcal{P}$ deja en N1\\
    $\mathcal{P}_2$ deja en N1 el valor que $\mathcal{P}$ deja en N2\\
    $\mathcal{P}_3$ deja en P1 el valor que $\mathcal{P}$ deja en P1\\
\end{center}
Aclaración 2:
\begin{center}
    $\Psi_E^{1, 0, \#}(x) = $ valor que representa N1 tras correr $E$ partiendo desde el estado $\llbracket x \rrbracket$\\
    $\Psi_E^{1, 0, *}(x) = $ valor que representa P1 tras correr $E$ partiendo desde el estado $\llbracket x \rrbracket$\\  
\end{center}
\bigskip

($\subseteq$) Para todo $t \in \omega$, por (2a) sabemos que $P$ partiendo de $\llbracket t \rrbracket$ llega a un estado de la forma $((x, y, z, \dots), (\alpha, \beta, \dots))$ donde $(x, y, \alpha) \in S$ y como $F = [F_1, F_2, F_3]$ y teniendo en cuenta ambas aclaraciones, $F(x) = (x, y, \alpha)$\\
\medskip

($\supseteq$) Sea $(x, y, \alpha) \in S$ sabemos que $\exists t \in \omega$ tal que correr $\mathcal{P}$ partiendo del estado $\llbracket t \rrbracket$ llega a un estado de la forma $((x, y, z, \dots), (\alpha, \beta, \dots))$. Entonces, por ambas aclaraciones, $(x, y, \alpha) = (F_{(1)}(t), F_{(2)}(t), F_{(3)}(t)) = F(t)$, luego, pertenece a $I_F$

\hfill $\blacksquare$

\bigskip
\bigskip
\bigskip

\textbf{Lemma 2} (Lema de la sumatoria). \textit{Sea $\Sigma$ un alfabeto finito. Si 
$f : \omega \times S_1 \times \cdots \times S_n \times L_1 \times \cdots \times L_m \to \omega$ 
es $\Sigma$-p.r., con $S_1, \ldots, S_n \subseteq \omega$ y $L_1, \ldots, L_m \subseteq \Sigma^*$ no vacíos, entonces la función
\[
\lambda x y \vec{x} \vec{\alpha} \left[\sum_{t=x}^{y} f(t, \vec{x}, \vec{\alpha})\right]
\]
es $\Sigma$-p.r.}
\bigskip

\textit{Proof.} Sea 
\[
G = \lambda t x \vec{x} \vec{\alpha} \left[\sum_{i=x}^{t} f(i, \vec{x}, \vec{\alpha}) \right]
\]
Ya que
\[
\lambda x y \vec{x} \vec{\alpha} \left[\sum_{i=x}^{y} f(i, \vec{x}, \vec{\alpha}) \right] 
= G \circ \left[p_2^{n+2,m}, p_1^{n+2,m}, p_3^{n+2,m}, \ldots, p_{n+m+2}^{n+2,m} \right]
\]
basta con probar que $G$ es $\Sigma$-p.r. Primero note que
\[
G(0, x, \vec{x}, \vec{\alpha}) = 
\begin{cases}
0 & \text{si } x > 0 \\
f(0, \vec{x}, \vec{\alpha}) & \text{si } x = 0
\end{cases}
\]
\[
G(t+1, x, \vec{x}, \vec{\alpha}) = 
\begin{cases}
0 & \text{si } x > t + 1 \\
G(t, x, \vec{x}, \vec{\alpha}) + f(t + 1, \vec{x}, \vec{\alpha}) & \text{si } x \leq t + 1
\end{cases}
\]

O sea que si definimos $h$ y $g$ apropiadamente, tenemos que $G = R(h, g)$. Definamos $h$ y $g$ y definamos los conjuntos $D_1, D_2, H_1, H_2$ basados en las condiciones de los casos.\\
\begin{align*}
    &D_1 = \{(x, \vec{x}, \vec{\alpha}) \in \omega \times S_1 \times \dots \times S_n \times L_1 \times \dots \times L_m : x > 0 \}\\
    &D_2 = \{(x, \vec{x}, \vec{\alpha}) \in \omega \times S_1 \times \dots \times S_n \times L_1 \times \dots \times L_m : x = 0 \}\\
    &H_1 = \{(z, t, x, \vec{x}, \vec{\alpha}) \in \omega^3 \times S_1 \times \dots \times S_n \times L_1 \times \dots \times L_m : x > t + 1 \}\\
    &H_2 = \{(z, t, x, \vec{x}, \vec{\alpha}) \in \omega^3 \times S_1 \times \dots \times S_n \times L_1 \times \dots \times L_m : x \leq t + 1 \}\\
\end{align*}

Ahora, $h$ y $g$ correspondientes:
\[
h = C_0^{n+1,m} \big|_{D_1} \cup \lambda x \vec{x} \vec{\alpha} \left[ f(0, \vec{x}, \vec{\alpha}) \right] \big|_{D_2}
\]

\[
g = C_0^{n+3,m} \big|_{H_1} \cup \lambda A t x \vec{x} \vec{\alpha} \left[A + f(t+1, \vec{x}, \vec{\alpha}) \right] \big|_{H_2}
\]

Ya que $f$ es $\Sigma$-p.r. y las funciones $\lambda x \vec{x} \vec{\alpha} \left[f(0, \vec{x}, \vec{\alpha}) \right]$ 
y $\lambda A t x \vec{x} \vec{\alpha} \left[A + f(t+1, \vec{x}, \vec{\alpha}) \right]$ se pueden escribir como composición 
de funciones $\Sigma$-p.r., estas también son $\Sigma$-p.r. Solo falta ver que los conjuntos $D_1, D_2, H_1, H_2$ son $\Sigma$-p.r. 
Veamos por ejemplo que $H_1$ es $\Sigma$-pr. Ya que $f$ es $\Sigma$-p.r., su dominio $D_f = \omega \times S_1 \times \cdots \times S_n \times L_1 \times \dots \times L_m$ es $\Sigma$-p.r., 
lo que implica que $S_i$ y $L_j$ son $\Sigma$-p.r. Por lo tanto, el conjunto 
$R = \omega^3 \times S_1 \times \cdots \times L_m$ es $\Sigma$-p.r. 
Nótese que $\chi_{H_1}^{\omega^{3+n} \times \Sigma^{*m}} = (\chi_{R}^{\omega^{3+n} \times \Sigma^{*m}} \wedge \lambda z t x \vec{x} \vec{\alpha} [x > t+1])$, y como es la conjunción de dos predicados $\Sigma$-p.r., es $\Sigma$-p.r. \hfill $\blacksquare$


\end{document}