\documentclass{article}
\usepackage[utf8]{inputenc}
\usepackage{amsfonts}
\usepackage{amsmath, amssymb}
\usepackage{mathrsfs}
\usepackage{enumitem}
\usepackage{stmaryrd}


\title{Lenguajes Formales y Computabilidad \\
        \large Teoremas: Combo 2 }

\author{Nicolás Cagliero}

\begin{document}
\maketitle

\textbf{Lema} (Lema de división por casos para funciones $\Sigma$-p.r.). 
\textit{Supongamos $f_i : D_{f_i} \subseteq \omega^n \times \Sigma^{*m} \to \Sigma^*$, $i = 1, ..., k$, son funciones $\Sigma$-p.r. tales que 
$D_{f_i} \cap D_{f_j} = \emptyset$ para $i \ne j$. Entonces $f_1 \cup \ldots \cup f_k$ es $\Sigma$-p.r.}

\medskip

(Hacer el caso $k = 2$, $n = 2$ y $m = 1$)

\medskip

\textit{Proof.} Supongamos $k = 2$. Sean
\[
\bar{f}_i : \omega^2 \times \Sigma^{*} \to \Sigma^*, \quad i = 1, 2,
\]
funciones $\Sigma$-p.r. tales que $\bar{f}_i|_{D_{f_i}} = f_i$, $i = 1, 2$ (Lema 19). Por la Proposición 20 los conjuntos $D_{f_1}$ y $D_{f_2}$ son $\Sigma$-p.r. y por lo tanto lo es $D_{f_1} \cup D_{f_2}$. Ya que

\[
f_1 \cup f_2 =
\left( \lambda \alpha \beta \left[ \alpha \beta \right] \circ 
\left[ \lambda x \alpha \left[ \alpha^x \right] \circ 
\left[ \chi_{D_{f_1}}^{\omega^2 \times \Sigma^{*}}, \bar{f}_1 \right], 
\lambda x \alpha \left[ \alpha^x \right] \circ 
\left[ \chi_{D_{f_2}}^{\omega^2 \times \Sigma^{*}}, \bar{f}_2 \right] \right] 
\right) \Big|_{D_{f_1} \cup D_{f_2}}
\]

tenemos que $f_1 \cup f_2$ es $\Sigma$-p.r. \hfill $\blacksquare$

\bigskip

\textbf{Proposición} (Caracterización básica de conjuntos $\Sigma$-enumerables). 
\textit{Sea $S \subseteq \omega^n \times \Sigma^{*m}$ un conjunto no vacío. Entonces son equivalentes:}

\begin{enumerate}
    \item $S$ es $\Sigma$-enumerable.
    \item Hay un programa $\mathcal{P} \in \text{Pro}^{\Sigma}$ tal que:
    \begin{enumerate}
        \item \textit{Para cada $x \in \omega$, tenemos que $\mathcal{P}$ se detiene partiendo desde el estado 
        $\llbracket x \rrbracket$ y llega a un estado de la forma $((x_1, ..., x_n, y_1, ...), (\alpha_1, ..., \alpha_m, \beta_1, ...))$, 
        donde $(x_1, ..., x_n, \alpha_1, ..., \alpha_m) \in S$.}
        
        \item \textit{Para cada $(x_1, ..., x_n, \alpha_1, ..., \alpha_m) \in S$ hay un $x \in \omega$ tal que $\mathcal{P}$ 
        se detiene partiendo desde el estado $\llbracket x \rrbracket$ y llega a un estado de la forma 
        $((x_1, ..., x_n, y_1, ...), (\alpha_1, ..., \alpha_m, \beta_1, ...))$.}
    \end{enumerate}
\end{enumerate}

\medskip

(Hacer el caso $n = 2$ y $m = 1$)
\medskip

\textit{Proof.} (1) $\Rightarrow$ (2). Ya que $S$ es no vacío, por definición existe una función 
\[
F : \omega \to \omega^2 \times \Sigma^*
\]
tal que $I_F = S$ y cada componente $F_{(i)}$ es $\Sigma$-computable, para $i = 1, 2, 3$.

Por el Primer Manantial de Macros, existen los siguientes macros:
\[
\begin{aligned}
&[\text{V2} \leftarrow F_{(1)}(\text{V1})] \\
&[\text{V2} \leftarrow F_{(2)}(\text{V1})] \\
&[\text{W1} \leftarrow F_{(3)}(\text{V1})]
\end{aligned}
\]

Sea $\mathcal{P}$ el siguiente programa:
\[
\begin{aligned}
&[\text{P}1 \leftarrow F_{(3)}(\text{N1})] \\
&[\text{N2} \leftarrow F_{(2)}(\text{N1})] \\
&[\text{N1} \leftarrow F_{(1)}(\text{N1})]
\end{aligned}
\]

donde se supone que las expansiones de los macros usan variables auxiliares que no aparecen en la lista $\text{N1}, \text{N2}, \text{P1}$ y tampoco se repiten labels auxiliares.\\
Corroborar que se cumplan las condiciones

(2) $\Rightarrow$ (1). Supongamos que $\mathcal{P} \in \text{Pro}^\Sigma$ cumple (a) y (b) de (2). Sea:

\[
\begin{aligned}
&\mathcal{P}_1 = \mathcal{P} \text{N1} \leftarrow \text{N1} \\
&\mathcal{P}_2 = \mathcal{P} \text{N1} \leftarrow \text{N2} \\
&\mathcal{P}_3 = \mathcal{P}\text{P1} \leftarrow \text{P2}
\end{aligned}
\]

Definimos entonces:
\[
\begin{aligned}
&F_1 = \Psi^{1,0,\#}_{\mathcal{P}_1} \\
&F_2 = \Psi^{1,0,\#}_{\mathcal{P}_2} \\
&F_3 = \Psi^{1,0,*}_{\mathcal{P}_3}
\end{aligned}
\]

Nótese que cada $F_i$ es $\Sigma$-computable y tiene dominio $\omega$. Sea $F = [F_1, F_2, F_3]$. Por definición, $D_F = \omega$ y como $F_{(i)} = F_i$ para $i = 1, 2, 3$, tenemos que cada componente de $F$ es $\Sigma$-computable.

Dejamos al lector verificar que $I_F = S$. \hfill $\blacksquare$

\end{document}