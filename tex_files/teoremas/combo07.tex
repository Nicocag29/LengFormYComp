\documentclass{article}
\usepackage[utf8]{inputenc}
\usepackage{amsfonts}
\usepackage{amsmath, amssymb}
\usepackage{mathrsfs}
\usepackage{enumitem}
\usepackage{stmaryrd}


\title{Lenguajes Formales y Computabilidad \\
        \large Teoremas: Combo 7 }

\author{Nicolás Cagliero}

\begin{document}
\maketitle

\textbf{Lema} (Lema de la minimización acotada). Sean \( n, m \geq 0 \). Sea 
\[
P : D_P \subseteq \omega \times \omega^n \times \Sigma^m \to \omega
\]
un predicado \( \Sigma \)-p.r. Entonces:

\begin{itemize}
    \item[(a)] \( M(P) \) es \( \Sigma \)-recursiva.
    \item[(b)] Si hay una función \( \Sigma \)-p.r. \( f : \omega^n \times \Sigma^m \to \omega \) tal que 
    \[
    M(P)(\vec{x}, \vec{\alpha}) = \text{min}_t P(t, \vec{x}, \vec{\alpha}) \leq f(\vec{x}, \vec{\alpha}), \quad \text{para cada } (\vec{x}, \vec{\alpha}) \in D_{M(P)},
    \]
    entonces \( M(P) \) es \( \Sigma \)-p.r.
\end{itemize}

\textbf{Dem} (a) Sea \( \widetilde{P} = P \cup C^{n+1,m}_0|_{[\omega^{n+1} \times \Sigma^{*m}] - D_P} \). Note que \( \widetilde{P} \) es \( \Sigma \)-p.r.. Veremos a continuación que \( M(P) = M(\widetilde{P}) \). Nótese que
\[
\{t \in \omega : P(t, \vec{x}, \vec{\alpha}) = 1\} = \{t \in \omega : \widetilde{P}(t, \vec{x}, \vec{\alpha}) = 1\}
\]
Esto claramente dice que \( D_{M(P)} = D_{M(\widetilde{P})} \) y que \( M(P)(\vec{x}, \vec{\alpha}) = M(\widetilde{P})(\vec{x}, \vec{\alpha}) \), para cada \( (\vec{x}, \vec{\alpha}) \in D_{M(P)} \), por lo cual \( M(P) = M(\widetilde{P}) \). Veremos entonces que \( M(\widetilde{P}) \) es \( \Sigma \)-recursiva. Sea \( k \) tal que \( \widetilde{P} \in PR^{\Sigma}_k \). Ya que \( \widetilde{P} \) es \( \Sigma \)-total y \( \widetilde{P} \in PR^{\Sigma}_k \subseteq R^{\Sigma}_k \), tenemos que \( M(\widetilde{P}) \in R^{\Sigma}_{k+1} \) y por lo tanto \( M(P) \in R^{\Sigma} \).

(b) Ya que \( M(P) = M(\widetilde{P}) \), basta con probar que \( M(\widetilde{P}) \) es \( \Sigma \)-p.r. Primero veremos que \( D_{M(\widetilde{P})} \) es un conjunto \( \Sigma \)-p.r. Nótese que
\[
\chi^{\omega^n \times \Sigma^m}_{D_{M(\widetilde{P})}} = \lambda \vec{x} \vec{\alpha} \left( [\exists t \in \omega]_{t \leq f(\vec{x}, \vec{\alpha})} \text{ } \widetilde{P}(t, \vec{x}, \vec{\alpha}) \right)
\]
lo cual nos dice que 
\[
\chi^{\omega^n \times \Sigma^m}_{D_{M(\widetilde{P})}} = \lambda x \vec{x} \vec{\alpha} \left( (\exists t \in \omega)_{t \leq x} \text{ } \widetilde{P}(t, \vec{x}, \vec{\alpha}) \right) \circ [f, p^{n, m}_1, \ldots, p^{n, m}_{n+m}]
\]

Pero el Lema de Cuantificación Acotada nos dice que \( \lambda x \vec{x} \vec{\alpha} \left[ (\exists t \in \omega)_{t \leq x} \text{ } \widetilde{P}(t, \vec{x}, \vec{\alpha}) \right] \) es \( \Sigma \)-p.r., por lo cual \( \chi^{\omega^n \times \Sigma^m}_{D_{M(\widetilde{P})}} \) lo es.

Sea
\[
P_1 = \lambda t \vec{x} \vec{\alpha} \left( \widetilde{P}(t, \vec{x}, \vec{\alpha}) \wedge (\forall j \in \omega)_{j \leq t} \text{ } j = t \vee \neg \widetilde{P}(j, \vec{x}, \vec{\alpha}) \right)
\]

Note que \( P_1 \) es \( \Sigma \)-total y \( \Sigma \)-p.r. Además, para \( (\vec{x}, \vec{\alpha}) \in \omega^n \times \Sigma^{*m} \) se tiene que \( P_1(t, \vec{x}, \vec{\alpha}) = 1 \) si y solo si \( (\vec{x}, \vec{\alpha}) \in D_{M(\widetilde{P})} \) y \( t = M(\widetilde{P})(\vec{x}, \vec{\alpha}) \). Esto nos dice que

\[
M(\widetilde{P}) = \left( \lambda \vec{x} \vec{\alpha} \left[  \prod_{t=0}^{f(\vec{x}, \vec{\alpha})} t^{P_1(t, \vec{x}, \vec{\alpha})} \right] \right) |_{D_{M(\widetilde{P})}}
\]

Solo resta probar que 
\[
F = \lambda \vec{x} \vec{\alpha} \left[  \prod_{t=0}^{f(\vec{x}, \vec{\alpha})} t^{P_1(t, \vec{x}, \vec{\alpha})} \right]
\]
lo es. Pero

\[
F = \lambda x y \vec{x} \vec{\alpha} \left[ \prod_{t=x}^{y} t^{P_1(t,\vec{x},\vec{\alpha})} \right] \circ [C^{n, m}_0, f, p^{n, m}_1, \ldots, p^{n, m}_{n+m}]
\]

y por el Lema de la Sumatoria, \( F \) es \( \Sigma \)-p.r. \hfill $\blacksquare$


\bigskip
\bigskip


\textbf{Lema} Supongamos $f : D_f \subseteq \omega^n \times \Sigma^{*m} \to O$ es $\Sigma$-recursiva y $S \subseteq D_f$ es $\Sigma$-r.e., entonces $f|_S$ es $\Sigma$-recursiva.
\bigskip


\textbf{Dem} Haremos el caso $n = m = 1$ y $O = \Sigma^*$. Tenemos que hay una función $F : \omega \to \omega \times \Sigma^*$ tal que $\text{Im}\, F = S$ y $F_{(1)}, F_{(2)}$ son $\Sigma$-recursivas. Por el Primer Manantial en $\mathcal{S}^\Sigma$ hay macros
\[
\begin{aligned}
&\text{[W2} \leftarrow f(\text{V1}, \text{W1})] \\
&\text{[V2} \leftarrow F_{(1)}(\text{V1})] \\
&\text{[W1} \leftarrow F_{(2)}(\text{V1})]
\end{aligned}
\]

Ya que los predicados $D = \lambda xy[x \neq y]$ y $D' = \lambda \alpha \beta[\alpha \neq \beta]$ son $\Sigma$-p.r., en $\mathcal{S}^\Sigma$ hay macros

\[
\begin{aligned}
&\text{[IF } D(\text{V1}, \text{V2}) \text{ GOTO A1]} \\
&\text{[IF } D'(\text{W1}, \text{W2}) \text{ GOTO A1]}
\end{aligned}
\]

Para hacer más amigable la lectura los escribiremos de la siguiente manera
\[
\begin{aligned}
&\text{[IF V1} \neq \text{V2 GOTO A1]} \\
&\text{[IF W1} \neq \text{W2 GOTO A1]}
\end{aligned}
\]

Sea $\mathcal{P}$ el siguiente programa:
\[
\begin{aligned}
\text{L2} &\quad [\text{N2} \leftarrow F_{(1)}(\text{N20})] \\
     &\quad [\text{P2} \leftarrow F_{(2)}(\text{N20})] \\
     &\quad [\text{IF N1} \neq \text{N2 GOTO L1}] \\
     &\quad [\text{IF P1} \neq \text{P2 GOTO L1}] \\
     &\quad [\text{P1} \leftarrow f(\text{N1}, \text{P1})] \\
     &\quad \text{GOTO L3} \\
\text{L1} &\quad \text{N20} \leftarrow \text{N20} + 1 \\
     &\quad \text{GOTO L2} \\
\text{L3} &\quad \text{SKIP}
\end{aligned}
\]

Es fácil ver que $\mathcal{P}$ computa a $f|_S$ \hfill $\blacksquare$


\end{document}