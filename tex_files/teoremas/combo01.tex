\documentclass{article}
\usepackage[utf8]{inputenc}
\usepackage{amsfonts}
\usepackage{amsmath, amssymb}
\usepackage{mathrsfs}
\usepackage{enumitem}


\title{Lenguajes Formales y Computabilidad \\
        \large Teoremas: Combo 1 }

\author{Nicolás Cagliero}

\begin{document}
\maketitle

\textbf{Proposición} (Caracterización de conjuntos $\Sigma$-p.r.). \textit{Un conjunto $S$ es $\Sigma$-p.r. sii $S$ es el dominio de alguna función $\Sigma$-p.r.}
\medskip

(En la inducción de la prueba hacer solo el caso de la composición)
\medskip

\textit{Proof.} $(\Rightarrow)$ Note que $S = D_{\text{Pred}_{S}^{\omega^n \times \Sigma^{*m}}}$.

$(\Leftarrow)$ Probaremos por inducción en $k$ que $D_F$ es $\Sigma$-p.r., para cada $F \in \text{PR}^{\Sigma}_{k}$.\\

El caso $k = 0$ es fácil, $F \in \{Suc, \text{ } Pred, \text{ } C_0^{0,0}, \text{ } C_{\varepsilon}^{0,0}\} \text{ } \cup \text{ } \{d_a : a \in \Sigma\} \text{ } \cup \text{ } \{p_j^{n,m} : 1 \leq j \leq n + m\}$ y los dominios de estas funciones son claramente $\Sigma$-p.r.. Supongamos el resultado vale para un $k$ fijo y supongamos $F \in \text{PR}^{\Sigma}_{k+1}$. Veremos entonces que $D_F$ es $\Sigma$-p.r. Hay varios casos.\\

Supongamos que $F = g \circ [g_1, ..., g_r]$ con $g, g_1, ..., g_r \in \text{PR}^{\Sigma}_k$. Si $F = \emptyset$, entonces es claro que $D_F = \emptyset$ es $\Sigma$-p.r. Supongamos entonces que $F$ no es la función vacía. Tenemos entonces que $r$ es de la forma $n + m$ y

\[
g : D_g \subseteq \omega^n \times \Sigma^{*m} \to O
\]

\[
g_i : D_{g_i} \subseteq \omega^k \times \Sigma^{*l} \to \omega, \quad i = 1, ..., n
\]

\[
g_i : D_{g_i} \subseteq \omega^k \times \Sigma^{*l} \to \Sigma^*, \quad i = n + 1, ..., n + m
\]

con $O \in \{\omega, \Sigma^*\}$ y $k, l \in \omega$. Por Lema 19 (Si $f : D_f \subseteq \omega^n \times \Sigma^{*m} \to O$ es $\Sigma$-pr. entonces existe una función $\Sigma$-pr, $\bar{f} : \omega^n \times \Sigma^{*m} \to O$ tal que $f = \bar{f}|_{D_f}$), hay funciones $\Sigma$-p.r. $\bar{g}_1, ..., \bar{g}_{n+m}$ las cuales son $\Sigma$-totales y cumplen

\[
g_i = \bar{g}_i|_{D_{g_i}}, \quad \text{para } i = 1, ..., n + m.
\]

Por hipótesis inductiva los conjuntos $D_g$, $D_{g_i}$, $i = 1, ..., n+m$, son $\Sigma$-p.r. y por lo tanto

\[
S = \bigcap_{i=1}^{n+m} D_{g_i}
\]

lo es. Nótese que

\[
\chi_{D_F}^{\omega^k \times \Sigma^l} = \left( \chi_{D_g}^{\omega^n \times \Sigma^{*m}} \circ [\bar{g}_1, ..., \bar{g}_{n+m}]  \land \chi_S^{\omega^k \times \Sigma^l}\right)
\]

lo cual nos dice que $D_F$ es $\Sigma$-p.r. \hfill $\blacksquare$

\bigskip

\textbf{Teorema} (Neumann vence a Gödel). \textit{Si $h$ es $\Sigma$-recursiva, entonces $h$ es $\Sigma$-computable}

\medskip

(En la inducción de la prueba hacer solo el caso $h = R(f, \mathcal{G})$, con $I_h \subseteq \omega$)
\medskip

\textit{proof.}
Probaremos por inducción en $k$ que
\[
(*) \quad \text{Si } h \in \text{R}_k^{\Sigma}, \text{ entonces } h \text{ es } \Sigma\text{-computable}.
\]

El caso $k = 0$: $h \in \{Suc, \text{ } Pred, \text{ } C_0^{0,0}, \text{ } C_{\varepsilon}^{0,0}\} \text{ } \cup \text{ } \{d_a : a \in \Sigma\} \text{ } \cup \text{ } \{p_j^{n,m} : 1 \leq j \leq n + m\}$. 
Los programas que computan cada función son:
\begin{center}
        $Pred = \Psi_{\mathcal{P}}^{1, 0, \#}$\\
        \medskip
        $Suc = \Psi_{\text{N1}\leftarrow\text{N1}+1}^{1,0,\#}$\\
        \medskip
        $C_0^{0,0} = \Psi_{\text{SKIP}}^{0,0,\#}$\\
        \medskip
        $C_{\varepsilon}^{0,0} = \Psi_{\text{SKIP}}^{0,0,*}$\\
        \medskip
        $d_a = \Psi_{\text{P1}\leftarrow\text{P1}.a}^{1,0,*}$\\
        \medskip
        $p_i^{n,m} = \Psi_{\text{N1}\leftarrow\text{N}\bar{i}}^{n, m, \#}$ para $i \in \{1, \dots, n\}$\\
        \medskip
        $p_i^{n,m} = \Psi_{\text{P1}\leftarrow\text{P}\overline{i - n}}^{n, m, *}$ para $i \in \{n + 1, \dots, n + m\}$\\
  

\end{center}

Supongamos que $(*)$ vale para $k$, veremos que vale para $k + 1$. Sea $h \in \text{R}_{k+1}^{\Sigma} - \text{R}_k^{\Sigma}$. Hay varios casos...

Supongamos $h = R(f, \mathcal{G})$, con
\[
f : S_1 \times \ldots \times S_n \times L_1 \times \ldots \times L_m \to \omega
\]
\[
g_a : \omega \times S_1 \times \ldots \times S_n \times L_1 \times \ldots \times L_m \times \Sigma^* \to \omega, \quad a \in \Sigma
\]

elementos de $\text{R}^\Sigma_k$. Sea $\Sigma = \{a_1, ..., a_r\}$. Por hipótesis inductiva, las funciones $f$, $g_a$, $a \in \Sigma$, son $\Sigma$-computables y por lo tanto tenemos macros

\[
[\text{V}\overline{n+1} \gets f(\text{V}1, ..., \text{V}\overline{n}, \text{W}1, ..., \text{W}\overline{m})]
\]
\[
[\text{V}\overline{n+2} \gets g_{a_i}(\text{V}1, ..., \text{V}\overline{n}, \text{V}\overline{n+1}, \text{W}1, ..., \text{W}\overline{m}, \text{W}\overline{m+1})], \quad i = 1, ..., r
\]

\bigskip

Podemos entonces hacer el siguiente programa:

\begin{align*}
\quad & [\text{N}\overline{n+1} \gets f(\text{N}1, ..., \text{N}\overline{n}, \text{P}1, ..., \text{P}\overline{m})] \\
\text{L}\overline{r+1} \text{ }& \text{IF P}\overline{m+1} \text{ BEGINS } a_1 \text{ GOTO L1} \\
& \vdots \\
& \text{IF P}\overline{m+1} \text{ BEGINS } a_r \text{ GOTO L}\overline{r} \\
& \text{GOTO L}\overline{r + 2} \\
\\
\text{L1} \quad & \text{P}\overline{m+1} \gets^\curvearrowright \text{P}\overline{m+1} \\
& [\text{N}\overline{n+1} \gets g_{a_1}(\text{N}\overline{n+1}, \text{N}1, ..., \text{N}\overline{n}, \text{P}1, ..., \text{P}\overline{m}, \text{W}\overline{m+2})] \\
& \text{P}\overline{m+2} \gets \text{P}\overline{m+2}. a_1 \\
& \text{GOTO L}\overline{r+1} \\
& \vdots \\
\\
\text{L}\overline{r} \quad & \text{P}\overline{m+1} \gets^\curvearrowright \text{P}\overline{m+1} \\
& [\text{N}\overline{n+1} \gets g_{a_r}(\text{N}\overline{n+1}, \text{N}1, ..., \text{N}\overline{n}, \text{P}1, ..., \text{P}\overline{m}, \text{P}\overline{m+2})] \\
& \text{P}m+2 \gets \text{P}m + 2 . a_r \\
& \text{GOTO L}\overline{r+1} \\
\\
\text{L}\overline{r+2} \quad & \text{N}1 \gets \text{N}\overline{n+1}
\end{align*}

Es fácil chequear que este programa computa $h$.  \hfill $\blacksquare$


\end{document}